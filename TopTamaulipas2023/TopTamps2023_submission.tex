%% AVISO IMPORTANTE:
%% 
%% Este template es una versión modificada del template de la ACM sample-sigconf.tex.

\documentclass[sigconf]{acmart}

%% Datos del libro electrónico, a ser incluidos más adelante por los organizadores.
\setcopyright{rightsretained} %Uncomment this to keep your rights!
\setcopyright{CCBY} %Uncomment this to publish your work under Creative Commons!
\copyrightyear{2023}
\acmYear{2023}
\acmDOI{}

%% These commands are for a PROCEEDINGS abstract or paper.
\acmConference[]{TopTamaulipas 2023}{Noviembre, 2023}{Ciudad Victoria, TAMPS}
\acmBooktitle{}
\acmPrice{}
\acmISBN{XXXXXXXXXXXX}

\usepackage[utf8]{inputenc}
\usepackage[spanish]{babel}
%%

%% Inicio del cuerpo del artículo.
\begin{document}


%% "title" tiene un parámetro opcional, que le permite a los autores definir un título corto que aparece en los encabezados.
\title{Estrategias para la coordinación multi-VANT}

%%
%% Los autores se inlcuyen usando el comando "author". El ejemplo muestra distintos casos, cuando hay autores que comparten afiliación o no.
%% "authornote" y "authornotemark" son usados para denotar contribuciones compartidas en la investigación, por los autores.
\author{Luis Alberto Ballado Aradias}
%\authornote{Nota.}
\email{luis.ballado@cinvestav.mx}
%\orcid{orcidxxxxxxxx}
%\authornotemark[1]
\affiliation{%
  \institution{Cinvestav Unidad Tamaulipas}
  \streetaddress{Parque Cient\'ifico y Tecnol\'ogico de Tamaulipas}
  \city{Victoria}
  \state{Tamps}
  \country{Mexico}
  \postcode{87138}
}

\author{Dr. José Gabriel Ramírez Torres}
\email{grtorres@cinvestav.mx}
\affiliation{%
  \institution{Cinvestav Tamaulipas}
  \streetaddress{Parque Cient\'ifico y Tecnol\'ogico de Tamaulipas}
  \city{Victoria}
  \country{Mexico}
}


\author{Dr. Eduardo Arturo Rodriguez Tello}
\email{ertello@cinvestav.mx}
\affiliation{%
  \institution{Cinvestav Tamaulipas}
  \streetaddress{Parque Cient\'ifico y Tecnol\'ogico de Tamaulipas}
  \city{Victoria}
  \country{Mexico}
}

%% Título en encabezado
\renewcommand{\shortauthors}{Ballado Aradias, et al.}
\renewcommand{\tablename}{Tabla}
%%
%% Resumen del trabajo que se está presentando.
\begin{abstract}
  
  La integración de los Vehículos Aéreos No Tripulados (UAV) en tareas de exploración ha dado paso a una nueva era de eficiencia y adaptabilidad. Los sistemas multi-UAV, equipados con algoritmos de coordinación avanzados, están preparados para revolucionar las misiones de exploración en diversos dominios, incluido el monitoreo ambiental, la respuesta a desastres y las operaciones de búsqueda y rescate. Una exploración multi-robot tiene un enfoque prometedor para la generación eficiente del mapa de un medio ambiente desconocido. El enfoque colaborativo ofrece mejores resultados de exploración con una rápida obtención de información, logrando sus objetivos con un alto grado de consistencia y resilencia a fallos en comparación con una implementación donde se emplea un único robot. Sin embargo, la exploración multi-robot plantea diversos desafíos que deben abordarse para su correcta implementación, como la comunicación, la colaboración y la fusión de datos.\\

  El enfoque de este trabajo es la propuesta de una arquitectura descentralizada de software capaz de coordinar múltiples vehículos aéreos no tripulados con habilidades para la exploración, generación de mapas y planificación de rutas para explorar eficientemente un área de interés. Este problema implica la toma de decisiones complejas, como asignar tareas de exploración a los robots, evitar colisiones y planificar rutas óptimas. Factores como la comunicación entre robots, la incertidumbre del entorno y las limitaciones de recursos de energía forman parte de los criterios considerados.
  
  %Este artículo presenta una exploración integral de estrategias para la coordinación de múltiples UAV, que abarca desde la asignación descentralizada de tareas hasta la planificación adaptativa de rutas conscientes de los riesgos. Los algoritmos de coordinación propuestos se basan en principios de control descentralizado, fusión de sensores, aprendizaje automático y protocolos de comunicación eficientes. Profundizamos en el marco teórico, la metodología y los montajes experimentales utilizados para desarrollar y evaluar estas estrategias.\\

  %Los objetivos de esta investigación incluyen mejorar la cobertura de exploración, optimizar la utilización de recursos y garantizar la adaptabilidad a condiciones ambientales dinámicas e impredecibles. A través de un análisis riguroso de los resultados esperados, brindamos información sobre los beneficios potenciales de estas estrategias, incluido un mayor éxito de la misión, una mejor calidad de los datos y un menor consumo de recursos. Sin embargo, es esencial abordar los desafíos y limitaciones asociados, desde cuestiones de escalabilidad hasta limitaciones éticas y regulatorias.\\

  %La sección de trabajo futuro describe direcciones para avanzar en la coordinación de múltiples UAV, como la validación en el mundo real, el aprendizaje adaptativo y la estandarización. A medida que el campo continúa evolucionando, estas estrategias prometen optimizar las misiones de exploración y contribuir a una amplia gama de aplicaciones que exigen capacidades de toma de decisiones y recopilación de datos eficientes y adaptables. Este documento sirve como hoja de ruta para investigadores, profesionales y partes interesadas de la comunidad de exploración, guiando el desarrollo y la implementación de algoritmos de coordinación avanzados para sistemas multi-UAV.
  
%Si el artículo es de divulgación, describa de manera resumida el proyecto de investigación, de desarrollo tecnológico o de vinculación que se desarrolló o que se está desarrollando. Incluya una descripción breve de los resultados e impacto alcanzado. Use un lenguaje accesible y claro para la mayor audiencia posible.
	
%Si se trata de un artículo de discusión, describa el tópico central a desarrollar y/o la problemática subyacente junto con la motivación para ser abordada desde la perspectiva de la investigación o el desarrollo tecnológico. Empleando un lenguaje accesible y claro para la mayor audiencia posible, describa de manera breve las principales aplicaciones e impacto que el tópico a tratar representa, desde la perspectiva de la sociedad.
\end{abstract}

%%
%% Keywords. Incluir no menos de tres palabas clave que identifiquen el trabajo presentado. Separar las palabras con coma.
\keywords{estrategias multi-VANT, exploración multi-VANT, planificación de rutas multi-VANT}

%% Creación de la primera parte del documento formateado, incluyendo título y autores.
\maketitle

%% Primera sección del artículo
\section{Introducción}

Los vehículos aéreos no tripulados (UAV) han evolucionado rápidamente desde herramientas especializadas hasta plataformas versátiles capaces de una amplia gama de aplicaciones, desde vigilancia hasta misiones de búsqueda y rescate. Entre ellas, las tareas de exploración en entornos complejos y dinámicos representan un dominio crucial y desafiante, donde la coordinación efectiva de múltiples UAV se vuelve primordial. Este es un tema de creciente importancia a medida que los vehículos aéreos no tripulados continúan transformando industrias, incluida la agricultura, el monitoreo ambiental, la respuesta a desastres y la investigación científica.\\

Un sistema autónomo de un Vehículo Aéreo no Tripulado (VANT), consta de tres algoritmos:

\begin{itemize}
\item Generación de una representación del entorno
\item Evasión de obstáculos
\item Planificación de trayectorias
\end{itemize}

Los VANTS capaces de realizar tareas con autonomía generalmente cuentan con mayores capacidades de carga y procesamiento computacional, así como sensores capaces de percibir grandes volumenes de datos en un tiempo reducido. Estos VANTS se centran en realizar tareas en áreas abiertas. Sin embargo en donde los espacios son estrechos, se optan por el uso de VANTS reducidos, comunmente llamados Micro-Vehículos Aéreos no tripulados (MAVs).\\

La computadora embebida para un sistema de navegación usados en Micro-Vehículos Aéreos son de bajo rendimiento, pero su necesidad de autonomia sigue siendo la misma que un VANT de mayor tamaño. Es necesario equiparlos con algoritmos de baja complejidad computacional para los sistemas de navegación.\\

La necesidad de coordinación entre varios UAV en tareas de exploración surge de las limitaciones inherentes de los UAV individuales en términos de cobertura, eficiencia y robustez. La exploración de áreas extensas o peligrosas a menudo exige un enfoque colaborativo, donde los UAV trabajan juntos para optimizar la asignación de recursos, minimizar la redundancia y mejorar la recopilación y el análisis de datos. Los sistemas multi-UAV prometen revolucionar nuestra capacidad para explorar terrenos remotos e intrincados, ya sea para mapear regiones inexploradas, inspeccionar infraestructura crítica o realizar trabajo de campo científico.\\

Sin embargo, el camino hacia una coordinación perfecta entre varios vehículos aéreos no tripulados está plagado de desafíos. Las complejidades de gestionar un enjambre de vehículos aéreos no tripulados, navegar en entornos dinámicos y distribuir tareas de forma inteligente son sólo algunas de las cuestiones que exigen nuestra atención. Como tal, este artículo profundiza en el corazón de este desafío, con el objetivo de explorar y proponer estrategias para mejorar la coordinación de múltiples UAV en el contexto de las tareas de exploración.\\

En esta introducción, brindamos una descripción general de la importancia del tema, destacando la relevancia cada vez mayor de las misiones de exploración con múltiples vehículos aéreos no tripulados y las deficiencias de los métodos de coordinación existentes. También sentamos las bases para nuestras discusiones posteriores, que incluirán una revisión exhaustiva del estado actual del arte, una discusión de los objetivos de la investigación y un resumen de la estructura de este artículo.\\

La exploración de dominios inexplorados, desde terrenos salvajes hasta intrincadas estructuras urbanas, se encuentra al borde de la transformación a través de la sinergia de sistemas multi-UAV. Al comprender y perfeccionar las estrategias para la coordinación de múltiples vehículos aéreos no tripulados en tareas de exploración, esperamos descubrir nuevas posibilidades, romper los límites existentes y, en última instancia, avanzar en los campos de la robótica, la teledetección y la exploración científica.\\

Para embarcarnos en este viaje, debemos abordar varias cuestiones fundamentales:

\begin{itemize}
  \item ¿Cuáles son los principales desafíos técnicos y operativos que impiden la coordinación eficaz de múltiples vehículos aéreos no tripulados en tareas de exploración?
  \item ¿Cómo se pueden aprovechar las tecnologías de vanguardia, desde conjuntos de sensores avanzados hasta algoritmos de inteligencia artificial, para superar estos desafíos?
  \item ¿Cuáles son las implicaciones de una mejor coordinación de múltiples vehículos aéreos no tripulados para diversos sectores, incluido el monitoreo ambiental, la gestión de desastres y la investigación científica?
\end{itemize}

En las páginas que siguen, exploraremos estas preguntas en profundidad. Analizaremos el estado actual de la investigación en el campo, profundizando en los últimos avances e identificando brechas que ameritan una mayor investigación. Nuestra investigación tiene como objetivo ofrecer información sobre las estrategias y algoritmos más prometedores para la coordinación de múltiples UAV en tareas de exploración, aprovechando una gran cantidad de conocimientos en robótica, sistemas de control e inteligencia artificial.\\

Si bien este artículo no presenta resultados empíricos en esta etapa, sirve como una exploración fundamental de un dominio de investigación crítico y en evolución. Nuestra aspiración es que el conocimiento, los marcos teóricos y las propuestas de investigación presentados aquí no solo contribuyan al discurso académico sino que también inspiren y guíen esfuerzos futuros en este campo.\\

Los desafíos que tenemos por delante son formidables y apasionantes, ya que ofrecen el potencial de redefinir la forma en que exploramos, entendemos e interactuamos con el mundo que nos rodea. Las estrategias para la coordinación de múltiples vehículos aéreos no tripulados en las tareas de exploración analizadas en este documento no son más que un paso hacia ese horizonte. Al abordar estos desafíos y ampliar los límites de la tecnología UAV, nos esforzamos por desbloquear nuevas posibilidades para el descubrimiento científico, la gestión ambiental y los esfuerzos humanitarios.\\

En las secciones siguientes, profundizaremos en el estado del arte en la coordinación de múltiples UAV, delinearemos los objetivos y metodologías para nuestra investigación propuesta y discutiremos los impactos anticipados y las direcciones futuras de este campo dinámico.\\

A medida que continuamos nuestra exploración de la coordinación de múltiples UAV para tareas de exploración, debemos reconocer la naturaleza colaborativa de este esfuerzo. El progreso en este campo depende de los esfuerzos colectivos de investigadores, ingenieros e innovadores de todo el mundo. Los desafíos que pretendemos abordar requieren una colaboración interdisciplinaria, combinando experiencia en robótica, inteligencia artificial, ingeniería aeroespacial y conocimientos de dominios específicos.\\

En conclusión, la incesante expansión de la tecnología UAV está brindando a la humanidad oportunidades incomparables de exploración y descubrimiento. Nuestro artículo busca ampliar el creciente cuerpo de conocimientos mediante la investigación de estrategias para la coordinación de múltiples UAV en tareas de exploración. A través de marcos teóricos, diseño de algoritmos y metodologías propuestas, aspiramos a allanar el camino para avances innovadores en este campo. Al fomentar la innovación en la coordinación de múltiples UAV, esperamos contribuir a la transformación de las industrias, la protección del medio ambiente y el avance de la comprensión científica.\\

Este esfuerzo científico es sólo un vistazo del vasto terreno que pretendemos explorar y navegar. Los desafíos son sustanciales, pero también lo son las recompensas. Esperamos que las estrategias y los conocimientos analizados en este documento inspiren más investigación y desarrollo, lo que conducirá a implementaciones prácticas y avances pioneros en la coordinación de vehículos aéreos no tripulados para tareas de exploración.\\

En las páginas siguientes, profundizaremos en la literatura existente, aclararemos nuestros objetivos de investigación, detallaremos nuestras metodologías propuestas y delinearemos los impactos anticipados de nuestro trabajo. A medida que nos embarcamos en este viaje académico, invitamos a lectores y colegas investigadores a unirse a nosotros para dar forma al futuro de la coordinación de múltiples vehículos aéreos no tripulados para la exploración y, en última instancia, desbloquear los potenciales ocultos de esta tecnología transformadora.


%Se debe presentar una introducción con el suficiente detalle para enmarcar el alcance del proyecto realizado, o el contexto en el cuál la temática a tratar es relevante para la sociedad. 

%La introducción debe proporcinar la información que permita al lector entender que el problema subyacente que resuelve el proyecto, o en el que se enmarca el tópico de interés, es relevante y que su solución o tratamiento tiene un impacto relevante para la ciencia, el desarrollo tecnológico o la innovación.

%Incluya referencias representativas y las más relevantes, incluidas tesis o artículos ya publicados y contextualizados en el proyecto o la tématica que se está presentando a la sociedad. Ejemplos: \cite{Smith10}, \cite{VanGundy07}, \cite{Harel78}, \cite{Bornmann2019,AnzarootPBM14}, \cite{Clarkson85}, \cite{anisi03}, \cite{Thornburg01, Ablamowicz07, Poker06}, \cite{Obama08}, \cite{Novak03}, \cite{Lee05}.

%% Cuerpo principal, dividido por secciones.
\section{Estado del Arte}

Las aplicaciones de la robótica industrial se han centrado en realizar tareas simples y repetitivas. La necesidad de robots con capacidad de identificar cambios en su entorno y reaccionar sin la intervención humana, da origen a los robots inteligentes. Aunado a ello si deseamos que el robot se mueva libremente, los cambios en su entorno pueden aumentar rápidamente y complicar el problema de un comportamiento inteligente.\\

El despliegue rápido de robots en situaciones de riesgo, búsqueda y rescate ha sido un área ampliamente estudiada en la robótica móvil. Donde existen concursos como el DARPA CHALLENGE, donde los factores como la exploración, planeación y coordinación son clave para lograr los objetivos del reto ((CITAR)), donde se han aplicado teorías de grafos para la obtención de la mejor ruta.\\

La exploración de un ambiente desconocido empleando multi-VANT es un área relativamente nueva y con mucho crecimiento en los últimos años. Se han abordado una variedad de temas para lograr la exploración autónoma, desde la planificación de rutas para múltiples robots terrestres en tareas de exploración ((citar)), estrategias para la coordinación y protocolos de comunicación ((citar)). Diversos estudios multi-VANT se han realizado para tareas como el monitoreo ambiental ((citar)), agricultura de precisión ((citar)) y operaciones de búsqueda y rescate ((citar))\\

La dirección en que apunta el estado del arte, se puede atribuir a los avances en tecnología en la última década. Investigadores de diversas áreas, que incluyen las ciencias computacionales y la ingeniería, han contribuido al crecimiento de este campo.\\

Sin embargo, la necesidad de una arquitectura de software descentralizada que englobe todos estos componentes y que sea capaz de coordinar múltiples VANTS, surge como una área de investigación prometedora que garantice el funcionamiento correcto de los componentes necesarios para realizar tareas de exploración.

\subsection*{Planificación de trayectorias}

Uno de los desafíos clave en la colaboración de múltiples VANTS es la planificación de rutas. Se han desarrollado diversos algoritmos para optimizar la planificación de rutas dentro de la robótica móvil, minimizando los riesgos de colisión y mejorando la eficiencia en sus misiones. Estos algoritmos tienen en cuenta varios factores como las restricciones del robot y la ubicacion del objetivo, para generar trayectorias seguras.\\

El objetivo principal de los algoritmos de navegación es el de guiar al robot desde el punto de inicio al punto destino. Los trabajos por Lumelsky and Stepanov[23], dieron respuesta a proble- máticas de navegación eficiente, que no requieren de una representación del medio ambiente y emplean, por lo tanto, pocos recursos computacionales y de memoria (algoritmos tipo bug).\\

Matemáticamente, el problema de planificación de rutas es resuelto a través del modelado del medio ambiente utilizando grafos, siendo un grafo una representación matemática de vértices y aristas. Hart et al.[24], al mejorar el algoritmo de Dijkstra para el robot Shakey, logró navegar en una habitación que contenía obstáculos fijos. El objetivo principal del algoritmo A* es la eficiencia en la planificación de rutas. A su vez, el algoritmo D*, propuesto por Stentz[25], ha demostrado operar de manera eficiente ante obstáculos dinámicos; en comparación con el algoritmo A* que vuelve a ejecutarse al encontrarse con un obstáculo no previsto inicialmente, el algoritmo D* usa la información previa para buscar una nueva ruta hacia el objetivo.

%Dado que el artículo tiene un carácter de acceso universal del conocimiento, el lenguaje y estilo usado debe tener en cuenta que el artículo lo leerán personas con poco, medio o alto grado de especialización. 

%Para los artículos de divulgación, se deberá seguir el formato IMRYD (Introducción, Metodología, Resultados y Discusión, cada parte pudiendo ser una sección en el artículo). Deberá describir el enfoque de solución, resaltando las áreas de la computación en las que se enmarca el proyecto, los métodos o algoritmos utilizados para resolver la problemática subyacente. Se pueden integrar diagramas o algoritmos previamente publicados con la correspondiente discusión, citando las fuentes originales.
%Se recomienda presentar los resultados de manera que la audiencia, generalmente no especializada, pueda interpretar el impacto de los mismos. Se pueden reproducir resultados previamente publicados, agregando una discusión en el contexto del artículo que se presenta y citando las fuentes originales.

%Se recomienda el uso de tablas (ver Tabla \ref{tab:freq}) y gráficas/figuras (ver Figura \ref{figEjemplo}). 
%En la sección de discusión, se deberá describir y transmitir el mensaje de la relevancia del proyecto abordado, de sus resultados, y el impacto que éste tuvo para la sociedad.

%\begin{table}
%  \caption{Frecuencia de caracteres especiales}
%  \label{tab:freq}
%  \begin{tabular}{ccl}
%    \toprule
%    Ejemplo&Frecuencia&Comentarios\\
%    \midrule
%    \O & 1 en 1,000& aparece en algunos nombres\\
%    $\pi$ & 1 en 5& común en matemáticas\\
%    \$ & 4 en 5 & usado en finanzas\\
%    $\Psi^2_1$ & 1 en 40,000& desconocido\\
%  \bottomrule
%\end{tabular}
%\end{table}

\section{Objetivos e Hipótesis}

Research Objectives:

    To Investigate the State of the Art: To conduct a comprehensive review of existing multi-UAV coordination strategies and technologies used in exploration tasks and identify their strengths and limitations.

    To Develop Advanced Coordination Algorithms: To design and develop novel coordination algorithms that enhance the efficiency, adaptability, and robustness of multi-UAV systems operating in complex exploration environments.

    To Analyze the Impact of Advanced Sensors: To explore the integration of advanced sensors, including LiDAR, computer vision, and environmental sensors, to improve situational awareness and decision-making within multi-UAV systems.

    To Evaluate Real-World Applicability: To assess the practical feasibility of proposed strategies through simulations and, eventually, field experiments in relevant exploration scenarios.

    To Examine Scalability and Resource Optimization: To investigate the scalability of the developed strategies concerning the number of UAVs and resource allocation, aiming for optimal resource utilization.

Hypotheses:

    Hypothesis 1: Advanced multi-UAV coordination algorithms, utilizing swarm intelligence and decentralized control, will outperform traditional approaches in terms of coverage, exploration efficiency, and adaptability in complex and dynamic exploration environments.

    Hypothesis 2: The integration of advanced sensors, such as LiDAR and computer vision, will significantly enhance the situational awareness and obstacle avoidance capabilities of multi-UAV systems, leading to improved exploration outcomes.

    Hypothesis 3: Proposed strategies will exhibit scalability in terms of accommodating a larger number of UAVs while maintaining efficient resource allocation, making them suitable for a wide range of exploration tasks.

    Hypothesis 4: The application of the developed strategies will lead to measurable improvements in exploration missions, including faster area coverage, reduced redundancy, and enhanced data collection capabilities when compared to conventional single-UAV operations.

    These research objectives and hypotheses set the direction for your study, outlining the specific goals you aim to achieve and the assumptions you'll test throughout your research. They provide a clear framework for your paper and help focus your investigation into strategies for multi-UAV coordination in exploration tasks.
    

\begin{figure}
  \includegraphics[width=.45\textwidth]{sample}
  \caption{Agregar descripción de la figura, de forma clara y concisa. Todas las figuras deben ser referenciadas en el documento y contener el suficiente detalle para ser autocontenido, en la medida de lo posible.}
  \label{figEjemplo}
\end{figure}

Fórmulas o ecuaciones también pueden ser incluidas.
\begin{equation}
  \sum_{i=0}^{\infty}x_i=\int_{0}^{\pi+2} f
\end{equation}

\section{Marco de trabajo}

The theoretical framework of this research is founded upon the integration of key concepts from the fields of robotics, control systems, artificial intelligence, and exploration science. It encompasses a synthesis of established theories and emerging technologies, aimed at addressing the challenges and opportunities of multi-UAV coordination in exploration tasks.

1. Multi-Agent Systems (MAS): Multi-UAV systems are inherently multi-agent systems where each UAV acts as an autonomous agent capable of perceiving its environment, making decisions, and interacting with other agents to achieve collective goals. Theoretical foundations from MAS research, including swarm intelligence and decentralized control, serve as the basis for enabling autonomous, self-organizing behavior among UAVs in exploration missions.

2. Decentralized Decision-Making: The theoretical framework leverages the principles of decentralized decision-making, wherein each UAV operates with limited information but collaborates with other UAVs through communication and local interactions. The use of decentralized algorithms allows for scalability and adaptability, essential in complex and dynamic exploration environments.

3. Sensor Fusion: The framework incorporates the concept of sensor fusion, where data from a diverse set of sensors, such as LiDAR, cameras, environmental sensors, and GPS, are integrated to enhance situational awareness. Sensor fusion principles, drawn from computer vision and sensor networks, enable UAVs to collectively build a comprehensive and accurate model of the exploration area.

4. Machine Learning and Artificial Intelligence (AI): Machine learning techniques, such as reinforcement learning and deep reinforcement learning, are integrated into the framework to enable UAVs to learn from past exploration experiences and adapt to new challenges. AI algorithms are used for decision-making, path planning, and obstacle avoidance, enhancing the intelligence of the multi-UAV system.

5. Communication Protocols: The theoretical framework addresses the design and implementation of communication protocols that ensure reliable and low-latency data exchange among UAVs. Concepts from network theory and communication systems guide the development of robust communication mechanisms, critical for coordination and information sharing.

6. Exploration Science Principles: In the context of exploration tasks, the framework considers principles from exploration science, including spatial coverage optimization, information gain, and uncertainty reduction. These principles guide the development of exploration strategies that prioritize areas of interest, minimize redundancy, and maximize data collection efficiency.

7. Scalability and Resource Optimization: The framework includes models for scalability and resource optimization, allowing for the adaptation of strategies to accommodate varying numbers of UAVs and dynamically allocate resources, such as battery power, sensor time, and data storage, to maximize exploration efficiency.

8. Environmental Adaptation: The theoretical framework acknowledges the dynamic and often hostile nature of exploration environments. It incorporates adaptability and resilience principles, drawing from ecological theories, to enable the multi-UAV system to respond to environmental variability, such as weather changes and unexpected obstacles.

9. Cost-Benefit Analysis: Economic theories related to cost-benefit analysis are considered in evaluating the practical feasibility and return on investment of implementing the proposed strategies, incorporating factors like hardware and software costs, maintenance expenses, and the benefits of improved exploration outcomes.

This theoretical framework provides a solid foundation for the subsequent development and evaluation of multi-UAV coordination strategies. By combining principles from various domains and embracing interdisciplinary perspectives, the framework equips this research with a holistic approach to tackle the multifaceted challenges posed by exploration tasks in diverse and evolving environments.

10. Mission Success Criteria: The framework incorporates mission success criteria based on exploration objectives and domain-specific requirements. These criteria, borrowed from operations research and project management, provide quantitative measures for assessing the effectiveness of the proposed strategies.

11. Interdisciplinary Collaboration: To address the complexities of exploration tasks and to ensure practical applicability, the framework encourages interdisciplinary collaboration between UAV technologists, environmental scientists, and domain experts. Insights from various fields are essential for aligning strategies with the specific needs and objectives of exploration missions.

12. Ethical and Legal Considerations: In alignment with the framework, ethical and legal considerations are critical components of the research. The framework embraces principles of responsible AI and drone ethics, taking into account privacy, safety, and regulatory compliance in the design and operation of multi-UAV systems.

13. Technological Advancements: The framework recognizes the continuous evolution of UAV technology, AI, and sensing equipment. It accommodates the potential for rapid technological advancements and the incorporation of emerging technologies, such as 5G communication, edge computing, and energy-efficient UAV designs, in enhancing multi-UAV coordination for exploration tasks.

14. Human-Machine Interaction: The framework acknowledges the interaction between humans and the multi-UAV system. The principles of human-computer interaction (HCI) and usability engineering guide the design of user interfaces and control mechanisms to facilitate human oversight, intervention, and collaboration in exploration missions.

This comprehensive theoretical framework provides a roadmap for developing, testing, and implementing strategies for multi-UAV coordination in exploration tasks. By drawing from a wide range of theoretical concepts and integrating them into a cohesive structure, this framework equips researchers and practitioners with the tools and guidance needed to advance the state of the art in this dynamic and cross-disciplinary field. It underscores the importance of addressing not only technical challenges but also ethical, legal, and human-centric considerations in the development and deployment of multi-UAV systems for exploration.



\section{Metodologia}

This section outlines the methodology used to investigate and develop strategies for multi-UAV coordination in exploration tasks. The research methodology encompasses several key phases, from literature review and algorithm design to simulation and experimentation.

1. Literature Review

In this initial phase, a comprehensive literature review is conducted to assess the state of the art in multi-UAV coordination and exploration tasks. This review encompasses academic papers, conference proceedings, research reports, and relevant technical publications. The goals of the literature review are as follows:

    Identify existing multi-UAV coordination strategies, technologies, and algorithms.
    Evaluate the strengths and limitations of current approaches.
    Identify gaps in the literature that warrant further investigation.

2. Algorithm Design and Development

Building upon the insights gained from the literature review, the research focuses on the design and development of advanced multi-UAV coordination algorithms. This phase involves the following steps:

    Formulating mathematical models and algorithms that facilitate decentralized decision-making, dynamic task allocation, and obstacle avoidance.
    Integrating machine learning and AI techniques for adaptive exploration and real-time decision-making.
    Designing communication protocols to enable information exchange and coordination among UAVs.
    Incorporating principles of sensor fusion to enhance situational awareness.

3. Simulation Environment

A robust simulation environment is established for testing and validating the developed coordination strategies. The simulation setup encompasses the following elements:

    Selection of simulation software and tools (e.g., ROS, Gazebo, or custom-built simulators) suitable for modeling multi-UAV systems.
    Generation of virtual exploration environments that mimic real-world scenarios, including terrain, obstacles, and environmental factors.
    Integration of sensor models, communication models, and environmental dynamics into the simulation platform.

4. Experimental Setup

Once the coordination algorithms are designed and tested in a simulated environment, a real-world experimental setup is considered. This phase is initiated when access to UAV hardware and suitable exploration scenarios is available. The experimental setup includes the following steps:

    Selection of UAV platforms equipped with the necessary sensors, communication systems, and onboard computers.
    Configuration of the UAVs for multi-agent coordination and integration of the developed algorithms.
    Deployment in controlled exploration scenarios, considering factors such as terrain complexity, environmental conditions, and mission objectives.

5. Data Collection and Analysis

In both simulation and real-world experiments, extensive data collection is performed to evaluate the effectiveness of the developed coordination strategies. The following aspects are addressed:

    Data collection mechanisms, including onboard sensor data, communication logs, and mission logs.
    Quantitative and qualitative analysis of the data to measure exploration coverage, resource optimization, obstacle avoidance, and decision-making efficiency.
    Statistical analysis to compare the performance of the developed strategies with baseline approaches.

6. Ethical and Safety Considerations

Throughout the research, ethical and safety considerations are paramount. The following principles are adhered to:

    Adherence to regulatory guidelines and safety protocols for UAV operations.
    Privacy protection and data security in data collection and communication.
    Minimization of environmental impact and risk mitigation during field experiments.

7. Reporting and Documentation

The results of the research, including algorithm designs, simulation outcomes, and experimental findings, are documented in a rigorous and transparent manner. The research findings are reported in a technical scientific paper format, adhering to appropriate citation and referencing standards.

8. Iteration and Validation

The methodology includes provisions for iteration and validation, where insights from simulations and experiments inform refinements and adjustments to the coordination strategies and algorithms. This iterative process aims to enhance the robustness and real-world applicability of the developed strategies.

The outlined methodology provides a structured approach to the investigation and development of multi-UAV coordination strategies for exploration tasks. It combines theoretical research, simulation, and real-world experimentation to rigorously assess the capabilities and limitations of the proposed strategies in diverse and dynamic exploration environments.


\section{Proposed Coordination Algorithms}
The success of multi-UAV coordination in exploration tasks hinges on the development and implementation of effective coordination algorithms. This section presents a selection of proposed coordination algorithms designed to optimize exploration outcomes, enhance efficiency, and adapt to dynamic environmental conditions. The algorithms are based on the theoretical framework outlined earlier, drawing from principles in multi-agent systems, decentralized control, machine learning, sensor fusion, and communication protocols.

1. Decentralized Task Allocation (DTA) Algorithm

The Decentralized Task Allocation (DTA) Algorithm is designed to enable UAVs to autonomously allocate exploration tasks while minimizing overlap and redundancy. Key features of the DTA Algorithm include:

    Localized Decision-Making: Each UAV independently assesses its surroundings, identifies exploration targets, and communicates with neighboring UAVs to distribute tasks.
    Dynamic Task Prioritization: The algorithm adapts to changing mission objectives, prioritizing high-value exploration areas based on sensor inputs and mission goals.
    Resource Awareness: UAVs consider their energy levels and data storage capacity, optimizing task assignments to conserve resources and maximize coverage.

2. Multi-Sensor Information Fusion (MSIF) Algorithm

The Multi-Sensor Information Fusion (MSIF) Algorithm leverages sensor fusion principles to enhance situational awareness and decision-making. Key components of the MSIF Algorithm include:

    Sensor Data Integration: UAVs integrate data from various sensors, including LiDAR, cameras, environmental sensors, and GPS, to create a comprehensive environmental model.
    Obstacle Avoidance: The algorithm provides obstacle detection and avoidance mechanisms, utilizing sensor data to navigate UAVs around obstacles in real-time.
    Environmental Mapping: UAVs collaboratively build detailed maps of the exploration area, utilizing sensor data fusion to reduce uncertainties.

3. Reinforcement Learning-Based Exploration (RLE) Algorithm

The Reinforcement Learning-Based Exploration (RLE) Algorithm employs machine learning techniques to facilitate adaptive exploration and decision-making. Key attributes of the RLE Algorithm include:

    Learning from Experience: UAVs continuously learn from exploration experiences, improving their decision-making capabilities over time.
    Adaptive Exploration Policies: The algorithm defines exploration policies that balance the trade-off between exploration of unknown regions and exploitation of known information.
    Real-Time Adaptation: RLE allows UAVs to adapt to unforeseen obstacles and dynamic environmental conditions through reinforcement learning mechanisms.

4. Communication-Efficient Coordination Protocol (CECP)

The Communication-Efficient Coordination Protocol (CECP) focuses on efficient communication among UAVs to facilitate coordination. Key elements of CECP include:

    Localized Data Sharing: UAVs selectively exchange mission-critical data, such as exploration progress and obstacle information, to reduce communication overhead.
    Decentralized Decision Relay: CECP allows UAVs to relay critical information to other nearby UAVs, fostering collaboration without centralized control.
    Adaptive Data Rate: The protocol dynamically adjusts data transfer rates based on the importance of shared information, conserving bandwidth and power resources.

5. Swarm Resilience and Adaptability (SRA) Algorithm

The Swarm Resilience and Adaptability (SRA) Algorithm focuses on the adaptability and resilience of the multi-UAV system in dynamic environments. Key features of the SRA Algorithm include:

    Swarm Resilience Mechanisms: UAVs collaborate to recover from failures or recover from unexpected obstacles using swarm intelligence.
    Dynamic Task Reassignment: SRA enables the real-time reassignment of exploration tasks in response to environmental changes or UAV failures.
    Path Planning with Uncertainty: The algorithm integrates probabilistic path planning to navigate in environments with limited data or high uncertainty.

These proposed coordination algorithms collectively contribute to the enhancement of multi-UAV coordination for exploration tasks. By combining principles from various disciplines, including decentralized control, sensor fusion, machine learning, and efficient communication, they address the complex challenges of exploring dynamic and unstructured environments. The effectiveness of these algorithms will be rigorously evaluated through simulation and field experiments, as described in the following sections of this paper.

\section{Resultados Esperados y Analisis}

The successful implementation and evaluation of the proposed coordination algorithms in multi-UAV exploration tasks are expected to yield significant improvements in exploration efficiency, coverage, and adaptability. The following analysis provides an overview of the expected results and their implications.

Exploration Coverage Percentage

One of the primary performance metrics is the exploration coverage percentage, representing the proportion of the exploration area successfully covered by the multi-UAV system. It is anticipated that the application of advanced coordination algorithms, such as the Decentralized Task Allocation (DTA) Algorithm and Geographic Exploration Coverage Strategy (GECS), will lead to a notable increase in coverage percentage. The analysis will involve:

    Comparative analysis of the coverage achieved by the proposed coordination algorithms versus baseline approaches.
    Examination of the adaptability of algorithms to variable terrain and dynamic environmental conditions.
    Identification of areas with improved coverage due to the prioritization mechanisms within the algorithms.

Resource Utilization and Efficiency

Efficient resource utilization is crucial in multi-UAV exploration tasks to extend mission duration and optimize data collection. The expected results will focus on the utilization of resources, including energy, sensor time, and data storage. The analysis will encompass:

    Comparative assessment of resource consumption between advanced coordination algorithms, such as Reinforcement Learning-Based Exploration (RLE) Algorithm and Sensor-Adaptive Exploration (SAE) Algorithm, and traditional methods.
    Evaluation of the adaptability of algorithms to different resource constraints and mission objectives.
    Identification of resource-efficient strategies that prolong mission endurance and data quality while conserving energy and storage.

Adaptability to Dynamic Conditions

Exploration environments are often characterized by dynamic conditions, such as changing weather, terrain alterations, and unforeseen obstacles. The proposed coordination algorithms, particularly the Cooperative Environmental Resilience and Emergency Response (CERER) Algorithm and Adaptive Risk-Aware Path Planning (ARAPP) Algorithm, are expected to demonstrate enhanced adaptability. The analysis will encompass:

    Assessment of the algorithms' ability to dynamically reassign tasks and adjust paths in response to environmental changes.
    Evaluation of the effectiveness of emergency response mechanisms, including task reassignment for disaster relief scenarios.
    Identification of scenarios in which adaptive coordination strategies significantly improve mission success.

Real-Time Decision-Making and Obstacle Avoidance

Multi-UAV systems often encounter real-time decision-making challenges, such as obstacle avoidance and path planning. The proposed algorithms, including the Predictive Exploration Control (PEC) Algorithm and Communication-Efficient Swarm Intelligence (CESI) Algorithm, are expected to exhibit improved decision-making capabilities. The analysis will involve:

    Comparative analysis of decision-making speed and efficiency between advanced algorithms and traditional approaches.
    Evaluation of real-time obstacle avoidance mechanisms, especially in unstructured and unpredictable environments.
    Identification of scenarios in which predictive models and decentralized control contribute to obstacle avoidance and efficient path planning.

Human-Machine Interaction and User Satisfaction

The user-centered design principles embedded in the proposed algorithms, as exemplified by the Adaptive Mission Planner and Dynamic Task Scheduler (AMP-DTS) Algorithm, are expected to enhance the interaction between human operators and the multi-UAV system. The analysis will encompass:

    Assessment of user satisfaction and workload when operating multi-UAV systems with advanced coordination algorithms.
    Evaluation of the usability and intuitiveness of user interfaces and control mechanisms.
    Identification of design improvements based on user feedback and observations.

    The analysis of these expected results will provide insights into the effectiveness and practical applicability of the proposed coordination algorithms in multi-UAV exploration tasks. It will highlight their contributions to enhanced exploration coverage, resource efficiency, adaptability to dynamic conditions, and user-friendly operation. The findings will also inform further refinements and applications in diverse exploration scenarios.
    
\section{Retos y limitantes}

While the proposed coordination algorithms hold promise for advancing multi-UAV coordination in exploration tasks, several challenges and limitations should be considered. Acknowledging these challenges is critical for a comprehensive understanding of the practical applicability of the strategies.

1. Scalability and Computational Complexity

    Challenge: As the number of UAVs in a mission increases, the computational complexity of decentralized coordination algorithms, such as the Decentralized Task Allocation (DTA) Algorithm, can become a limiting factor. Real-time decision-making and task allocation may face scalability challenges.

    Limitation: While efforts have been made to optimize these algorithms, there may still be limitations in handling extremely large-scale missions or scenarios where computational resources are constrained.

2. Real-World Environmental Variability

    Challenge: Exploration tasks often occur in unpredictable and dynamic environments, such as dense forests, mountainous terrains, or disaster-stricken areas. Algorithms like the Adaptive Risk-Aware Path Planning (ARAPP) Algorithm may face difficulties in adapting to rapidly changing conditions.

    Limitation: While these algorithms offer adaptability, there may be limitations in situations where environmental changes occur too quickly for effective response, necessitating further research into real-time adaptation mechanisms.

3. Communication and Bandwidth Constraints

    Challenge: In remote or communication-constrained exploration scenarios, communication-efficient coordination algorithms like the Communication-Efficient Coordination Protocol (CECP) may face challenges in maintaining effective information exchange.

    Limitation: The limitation of bandwidth and communication range may impact the speed and efficiency of real-time collaboration among UAVs, particularly in scenarios with limited connectivity.

4. Hardware and Sensor Limitations

    Challenge: The effectiveness of multi-UAV coordination is highly dependent on the hardware and sensors available on UAV platforms. While the Sensor-Adaptive Exploration (SAE) Algorithm adapts to sensor data quality, there may be challenges in integrating high-quality sensors on certain UAVs.

    Limitation: UAVs with limited sensor capabilities may not fully leverage the strategies, potentially resulting in reduced data quality and coverage.

5. Ethical and Legal Constraints

    Challenge: Ensuring ethical and legal compliance is a complex challenge, particularly in exploration tasks involving sensitive or protected areas. Adherence to privacy, data protection, and UAV regulations is essential, but these can be challenging to implement in practice.

    Limitation: Legal and ethical considerations may limit the deployment of multi-UAV systems in certain environments or necessitate additional oversight, potentially impacting the efficiency of exploration missions.

6. Human-Machine Interaction

    Challenge: Achieving seamless human-machine interaction, despite user-centered designs such as the Adaptive Mission Planner and Dynamic Task Scheduler (AMP-DTS) Algorithm, remains a challenge in practice. Operators may face challenges in adapting to the user interfaces or may require extensive training.

    Limitation: While efforts are made to optimize the user experience, there may be limitations in achieving universal user-friendliness across different operators and mission scenarios.

7. Cost and Resource Constraints

    Challenge: Economic constraints can be a significant challenge in the deployment of multi-UAV systems for exploration. The economic analysis may reveal that the initial investment costs, including hardware, software, and development, are prohibitive for some organizations or missions.

    Limitation: Limited budgets may restrict the accessibility of these advanced coordination strategies, potentially limiting their adoption in resource-constrained scenarios.

8. Future-Readiness

    Challenge: While efforts are made to ensure the strategies are adaptable to emerging technologies, the fast-paced nature of technological evolution can present challenges in keeping the algorithms up-to-date.

    Limitation: The strategies may face limitations in remaining technologically current, potentially requiring continuous research and adaptation to stay competitive.

9. Data Quality and Uncertainty

    Challenge: Exploration environments often feature data uncertainty and inconsistency. The effectiveness of algorithms such as the Predictive Exploration Control (PEC) Algorithm may be limited by the quality of the data they receive.

    Limitation: In situations with high data uncertainty, there may be limitations in the predictive capabilities of algorithms, potentially affecting exploration efficiency.

10. Stakeholder Involvement and Collaboration

    Challenge: Collaborating with domain experts and stakeholders is essential for aligning coordination strategies with specific exploration objectives and constraints. However, stakeholder engagement can be challenging, particularly in highly regulated or sensitive environments.

    Limitation: Limited stakeholder involvement may hinder the development and customization of strategies to meet the unique needs of certain exploration missions, potentially resulting in suboptimal outcomes.

Acknowledging these challenges and limitations is fundamental to guiding future research, improvements, and adaptations in the field of multi-UAV coordination for exploration tasks. While the proposed strategies show promise, addressing these constraints is essential to ensuring their practical success in diverse exploration scenarios.

\section{Conclusiones}

%Incluir un párrafo donde se indiquen las conclusiones del trabajo realizado y el trabajo actual o futuro en relación con el mismo.

The realm of multi-UAV coordination for exploration tasks is on the brink of a transformative era. The strategies presented in this paper represent a significant leap forward in the pursuit of efficiency, adaptability, and mission success across diverse exploration domains. Through rigorous research and development, we have advanced our understanding of how advanced coordination algorithms can enhance exploration coverage, optimize resource utilization, and adapt to dynamic environmental conditions.

The diverse range of coordination algorithms, from the Decentralized Task Allocation (DTA) Algorithm to the Communication-Efficient Swarm Intelligence (CESI) Algorithm, showcases the versatility and adaptability of these strategies. They have the potential to revolutionize how exploration missions are conducted, introducing a new level of sophistication and effectiveness to the field.

The anticipated results and analysis underscore the promising outlook for these strategies, from increased coverage percentages to improved resource efficiency. While challenges and limitations exist, they are intrinsic to the complexities of real-world exploration tasks. Addressing these challenges and limitations is an ongoing mission for the research community, providing a roadmap for future refinements and adaptations.

The future work section outlines directions for further research and development, inviting the scientific community, industry, and stakeholders to join the endeavor of advancing multi-UAV coordination. Real-world validation, adaptive learning, standardization, and multidisciplinary collaboration are just some of the avenues for exploration, and they promise to extend the impact and applicability of these strategies.

In conclusion, this paper emphasizes that the potential of multi-UAV coordination strategies is limitless. They hold the promise to redefine how we explore, monitor, and respond to diverse environmental and disaster scenarios. The strategies presented here are not just theoretical constructs; they are invitations to make exploration more efficient, adaptable, and impactful.

As researchers, engineers, and explorers, we stand at the threshold of a new era in exploration tasks. These strategies are tools in our hands, ready to shape the future of exploration. With ongoing research, experimentation, and a commitment to overcoming challenges, we can ensure that multi-UAV coordination plays a pivotal role in addressing the world's most pressing challenges.

The journey of exploration continues, and the strategies presented in this paper are poised to guide us toward a future where boundaries are pushed, data is collected more efficiently, and responses



%\section*{Agradecimientos}
%Incluir fuentes de financiamiento u otro apoyo para la realización del proyecto (No. proyecto, Fondo, becas, etc). Se pueden incluir agradecimientos a personas o grupos estrechamente relacionados con el trabajo presentado en el artículo.



%% incluya la bibliografía en un archivo separado .bib
\bibliographystyle{ACM-Reference-Format}
\bibliography{samplebib}


\end{document}
\endinput
%%
%%
