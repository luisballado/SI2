%\documentclass[11pt,epsf,times,twocolumn]{article}
\documentclass[11pt,epsf,times]{article}

\usepackage{graphicx}
\usepackage{tikz}
\usepackage{forest}
\usetikzlibrary{shadows,arrows.meta}

\tikzset{parent/.style={align=center,text width=2cm,fill=green!20,rounded corners=2pt},
    child/.style={align=center,text width=2.8cm,fill=green!50,rounded corners=6pt},
    grandchild/.style={fill=pink!50,text width=2.3cm}
}

\usepackage{epsf,latexsym}
\usepackage[spanish,mexico]{babel}
\usepackage[latin1]{inputenc}

\hoffset=-9pt %-19pt
\voffset=-36pt
\textwidth=505pt
\marginparsep=30pt
\columnsep=9.9mm

\def\figurename{Figura}

\textwidth 6.5in
\textheight 8.5in %ancho en footer
\oddsidemargin 0in
\evensidemargin 0in
\topmargin -0.5in

%-----------------------

\title{ Centro de Investigaci\'{o}n y Estudios Avanzados del IPN\\
  Unidad Tamaulipas\\
  \textbf{Propuesta Top Tamaulipas 2023}
}

\author{
T\'{i}tulo: Estrategias para la exploraci\'{o}n coordinada multi-VANT \\
Alumno: Luis Alberto Ballado Aradias \\
Asesor: Dr. Jos\'{e} Gabriel Ram\'{i}rez Torres \\
Co-Asesor: Dr. Eduardo Arturo Rodr\'{i}guez Tello
}


\date{\today}

\usepackage{dirtree}
\usepackage{amssymb}

\usepackage{hyperref}
\usepackage{dirtree}
\usepackage{xcolor,colortbl}
\usepackage{multirow}

\usepackage{bbding} %palomitas checkmark
\usepackage{pifont}

% A package which allows simple repetition counts, and some useful commands

\usepackage{graphicx}
\usepackage{tabularx,booktabs}

\newcolumntype{C}{>{\centering\arraybackslash}X}
\setlength{\extrarowheight}{1pt}

\usepackage{multicol}

\usepackage[scaled]{helvet}
\usepackage[authoryear]{natbib}

\usepackage[ruled,vlined]{algorithm2e}

\usepackage{array} % needed for \arraybackslash
\usepackage{adjustbox} % for \adjincludegraphics
\usepackage{enumitem}% http://ctan.org/pkg/enumitem
\usepackage{forloop}
\usepackage{pdflscape}
\newcounter{loopcntr}

\let\oldfootnote\footnote

\renewcommand{\footnote}[1]{%
    \oldfootnote{\rlap{\parbox{\dimexpr\paperwidth-72pt\relax}{#1}}}%
}

\newcommand{\rpt}[2][1]{%
  \forloop{loopcntr}{0}{\value{loopcntr}<#1}{#2}%
}
\newcommand{\on}[1][1]{
  \forloop{loopcntr}{0}{\value{loopcntr}<#1}{&\cellcolor{gray}}
}
\newcommand{\off}[1][1]{
  \forloop{loopcntr}{0}{\value{loopcntr}<#1}{&}
}

\addtolength{\textheight}{90pt}

\newcommand{\I}{\mathbb{I}}
\newcommand{\K}{\mathbb{K}}
\newcommand{\N}{\mathbb{N}}
\newcommand{\Q}{\mathbb{Q}}
\newcommand{\R}{\mathbb{R}}
\newcommand{\Z}{\mathbb{Z}}

\newcommand{\specialcell}[2][c]{%
  \begin{tabular}[#1]{@{}c@{}}#2\end{tabular}}

%---------------COSAS DEL TIMELINE --------
\usepackage[T1]{fontenc}
\usepackage{lipsum}
\usepackage{charter}
\usepackage{environ}
\usetikzlibrary{calc,matrix}
\usepackage{soul}

\makeatletter
\let\matamp=&
\catcode`\&=13
\makeatletter
\def&{\iftikz@is@matrix
  \pgfmatrixnextcell
  \else
  \matamp
  \fi}
\makeatother

\newcounter{lines}
\def\endlr{\stepcounter{lines}\\}

\newcounter{vtml}
\setcounter{vtml}{0}

\newif\ifvtimelinetitle
\newif\ifvtimebottomline
\tikzset{description/.style={
  column 2/.append style={#1}
 },
 timeline color/.store in=\vtmlcolor,
 timeline color=red!80!black,
 timeline color st/.style={fill=\vtmlcolor,draw=\vtmlcolor},
 use timeline header/.is if=vtimelinetitle,
 use timeline header=false,
 add bottom line/.is if=vtimebottomline,
 add bottom line=false,
 timeline title/.store in=\vtimelinetitle,
 timeline title={},
 line offset/.store in=\lineoffset,
 line offset=4pt,
}

\NewEnviron{vtimeline}[1][]{%
\setcounter{lines}{1}%
\stepcounter{vtml}%
\begin{tikzpicture}[column 1/.style={anchor=east},
 column 2/.style={anchor=west},
 text depth=0pt,text height=1ex,
 row sep=1ex,
 column sep=1em,
 #1
]
\matrix(vtimeline\thevtml)[matrix of nodes]{\BODY};
\pgfmathtruncatemacro\endmtx{\thelines-1}
\path[timeline color st] 
($(vtimeline\thevtml-1-1.north east)!0.5!(vtimeline\thevtml-1-2.north west)$)--
($(vtimeline\thevtml-\endmtx-1.south east)!0.5!(vtimeline\thevtml-\endmtx-2.south west)$);
\foreach \x in {1,...,\endmtx}{
 \node[circle,timeline color st, inner sep=0.15pt, draw=white, thick] 
 (vtimeline\thevtml-c-\x) at 
 ($(vtimeline\thevtml-\x-1.east)!0.5!(vtimeline\thevtml-\x-2.west)$){};
 \draw[timeline color st](vtimeline\thevtml-c-\x.west)--++(-3pt,0);
 }
 \ifvtimelinetitle%
  \draw[timeline color st]([yshift=\lineoffset]vtimeline\thevtml.north west)--
  ([yshift=\lineoffset]vtimeline\thevtml.north east);
  \node[anchor=west,yshift=16pt,font=\large]
   at (vtimeline\thevtml-1-1.north west) 
   {\textsc{Linea de tiempo}: \textit{\vtimelinetitle}};
 \else%
  \relax%
 \fi%
 \ifvtimebottomline%
   \draw[timeline color st]([yshift=-\lineoffset]vtimeline\thevtml.south west)--
  ([yshift=-\lineoffset]vtimeline\thevtml.south east);
 \else%
   \relax%
 \fi%
\end{tikzpicture}
}
%---------------COSAS DEL TIMELINE --------

\begin{document}

\maketitle
\begin{abstract}

  Los veh�culos a�reos no tripulados (VANTS) o drones se han vuelto m�s frecuentes en el mundo actual, la versatilidad y adaptabilidad, los hacen indispensables en las operaciones modernas en numerosos sectores, ofreciendo soluciones eficientes, rentables e innovadoras. Al combinar tecnolog�as avanzadas y capacidades de toma de decisiones inteligentes con su capacidad para navegar de forma aut�noma, evitar obst�culos y llevar a cabo misiones complejas, los VANTS aut�nomos ofrecen mayor eficiencia, reducci�n de costos y mayor seguridad en una amplia gama de aplicaciones.\\

  No se puede subestimar la importancia de utilizar m�ltiples VANTS en actividades de exploraci�n, los sistemas multi-VANT permiten la ejecuci�n paralela de tareas, lo que reduce el tiempo de exploraci�n. Los VANTS pueden colaborar, intercambiar informaci�n y optimizar sus rutas para minimizar la redundancia y agilizar el proceso de exploraci�n. Si un VANT encuentra dificultades, otros VANTS pueden continuar la exploraci�n sin problemas, asegurando la continuidad de la misi�n y reduciendo el riesgo de fracaso de la misi�n. La colaboraci�n y coordinaci�n entre robots permite tiempos de respuesta m�s r�pidos, una cobertura m�s amplia de las �reas afectadas, lo que en �ltima instancia salva vidas.\\
  
  Las exploraciones con m�ltiples robots traen consigo complejidades inherentes que deben abordarse cuidadosamente. La coordinaci�n y colaboraci�n entre m�ltiples robots presenta desaf�os en t�rminos de comunicaci�n, asignaci�n de tareas y sincronizaci�n. Establecer arquitecturas de software y canales de comunicaci�n efectivos entre los robots es crucial para compartir informaci�n, coordinar acciones y evitar conflictos. Los VANTS del ma�ana deber�n navegar en �reas urbanas de la mejor manera posible y tener la habilidad de trabajar con m�ltiples agentes.\\  

  \noindent \textbf{Palabras claves:} exploraci\'{o}n multi-VANT, planificaci\'{o}n de rutas multi-VANT, arquitectura de software multi-VANT.
  
\end{abstract}

\end{document}
